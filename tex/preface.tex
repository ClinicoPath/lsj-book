% !TEX root = ../pdf/lsj.tex
% [There are multiple lsj.tex files, but the one in ../pdf is the usual one]


\addcontentsline{toc}{chapter}{Preface}
\newcommand{\vsp}{\vspace*{6pt}}

\begin{center}{\Large {\bf Preface to Version 0.70}}\end{center}
\vspace*{12pt}

\noindent
This update from version 0.65 introduces some new analyses. In the ANOVA chapters we have added sections on repeated measures ANOVA and analysis of covariance (ANCOVA). In a new chapter we have introduced Factor Analysis and related techniques. Hopefully the style of this new material is consistent with the rest of the book, though eagle-eyed readers might spot a bit more of an emphasis on conceptual and practical explanations, and a bit less algebra. I'm not sure this is a good thing, and might add the algebra in a bit later. But it reflects both my approach to understanding and teaching statistics, and also some feedback I have received from students on a course I teach. In line with this, I have also been through the rest of the book and tried to separate out some of the algebra by putting it into a box or frame. It's not that this stuff is not important or useful, but for some students they may wish to skip over it and therefore the boxing of these parts should help some readers. 

With this version I am very grateful to comments and feedback received from my students and colleagues, notably Wakefield Morys-Carter, and also to numerous people all over the world who have sent in small suggestions and corrections - much appreciated, and keep them coming! One pretty neat new feature is that the example data files for the book can now be loaded into jamovi as an add-on module - thanks to Jonathon Love for helping with that.

\vspace*{24pt}
\noindent
David Foxcroft 

\noindent
February 1st, 2019

\vspace*{30pt}


\begin{center}{\Large {\bf Preface to Version 0.65}}\end{center}
\vspace*{12pt}

\noindent
In this adaptation of the excellent `Learning statistics with R', by Danielle Navarro, we have replaced the statistical software used for the analyses and examples with jamovi. Although R is a powerful statistical programming language, it is not the first choice for every instructor and student at the beginning of their statistical learning. Some instructors and students tend to prefer the point-and-click style of software, and that's where jamovi comes in. jamovi is software that aims to simplify two aspects of using R. It offers a point-and-click graphical user interface (GUI), and it also provides functions that combine the capabilities of many others, bringing a more SPSS- or SAS-like method of programming to R. Importantly, jamovi will always be free and open - that's one of its core values - because jamovi is made by the scientific community, for the scientific community.

With this version I am very grateful for the help of others who have read through drafts and provided excellent suggestions and corrections, particularly Dr David Emery and Kirsty Walter.

\vspace*{24pt}
\noindent
David Foxcroft 

\noindent
July 1st, 2018

\vspace*{35pt}


\begin{center}{\Large {\bf Preface to Version 0.6}}\end{center}
\vspace*{12pt}

\noindent
The book hasn't changed much since 2015 when I released Version 0.5 -- it's probably fair to say that I've changed more than it has. I moved from Adelaide to Sydney in 2016 and my teaching profile at UNSW is different to what it was at Adelaide, and I haven't really had a chance to work on it since arriving here! It's a little strange looking back at this actually. A few quick comments...

\begin{itemize}
\item Weirdly, the book {\it consistently} misgenders me, but I suppose I have only myself to blame for that one :-) There's now a brief footnote on page 12 that mentions this issue; in real life I've been working through a gender affirmation process for the last two years and mostly go by she/her pronouns. I am, however, just as lazy as I ever was so I haven't bothered updating the text in the book.  
\item For Version 0.6 I haven't changed much I've made a few minor changes when people have pointed out typos or other errors. In particular it's worth noting the issue associated with the etaSquared function in the {\bf lsr} package (which isn't really being maintained any more) in Section 14.4. The function works fine for the simple examples in the book, but there are definitely bugs in there that I haven't found time to check! So please take care with that one. 
\item The biggest change really is the licensing! I've released it under a Creative Commons licence (CC BY-SA 4.0, specifically), and placed all the source files to the associated GitHub repository, if anyone wants to adapt it.
\end{itemize} 

\noindent
Maybe someone would like to write a version that makes use of the {\bf tidyverse}... I hear that's become rather important to R these days :-)

\vspace*{24pt}
\noindent
Best,

\noindent
Danielle Navarro

 
\vspace*{30pt}

\begin{center}{\Large {\bf Preface to Version 0.5}}\end{center}
\vspace*{12pt}

\noindent
Another year, another update. This time around, the update has focused almost entirely on the theory sections of the book. Chapters 9, 10 and 11 have been rewritten, hopefully for the better. Along the same lines, Chapter 17 is entirely new, and focuses on Bayesian statistics. I think the changes have improved the book a great deal. I've always felt uncomfortable about the fact that all the inferential statistics in the book are presented from an orthodox perspective, even though I almost always present Bayesian data analyses in my own work. Now that I've managed to squeeze Bayesian methods into the book somewhere, I'm starting to feel better about the book as a whole. I wanted to get a few other things done in this update, but as usual I'm running into teaching deadlines, so the update has to go out the way it is!

\vspace*{24pt}
\noindent
Dan Navarro 

\noindent
February 16, 2015

\vspace*{30pt}

\begin{center}{\Large {\bf Preface to Version 0.4}}\end{center}
\vspace*{12pt}

\noindent
A year has gone by since I wrote the last preface. The book has changed in a few important ways: Chapters 3 and 4 do a better job of documenting some of the time saving features of Rstudio, Chapters 12 and 13 now make use of new functions in the lsr package for running chi-square tests and t tests, and the discussion of correlations has been adapted to refer to the new functions in the lsr package. The soft copy of 0.4 now has better internal referencing (i.e., actual hyperlinks between sections), though that was introduced in 0.3.1. There's a few tweaks here and there, and many typo corrections (thank you to everyone who pointed out typos!), but overall 0.4 isn't massively different from 0.3. 

I wish I'd had more time over the last 12 months to add more content. The absence of any discussion of repeated measures ANOVA and mixed models more generally really does annoy me. My excuse for this lack of progress is that my second child was born at the start of 2013, and so I spent most of last year just trying to keep my head above water. As a consequence, unpaid side projects like this book got sidelined in favour of things that actually pay my salary! Things are a little calmer now, so with any luck version 0.5 will be a bigger step forward.

One thing that has surprised me is the number of downloads the book gets. I finally got some basic tracking information from the website a couple of months ago, and (after excluding obvious robots) the book has been averaging about 90 downloads per day. That's encouraging: there's at least a few people who find the book useful!


\vspace*{24pt}
\noindent
Dan Navarro 

\noindent
February 4, 2014


\vspace*{30pt}



%\addtocontents{toc}{\protect\setcounter{tocdepth}{-1}}
\begin{center}{\Large {\bf Preface to Version 0.3}}\end{center}
\vspace*{12pt}


\noindent
There's a part of me that really \underline{doesn't} want to publish this book. It's not finished. \vsp

%\noindent
And when I say that, I mean it. The referencing is spotty at best, the chapter summaries are just lists of section titles, there's no index, there are no exercises for the reader, the organisation is suboptimal, and the coverage of topics is just not comprehensive enough for my liking. Additionally, there are sections with content that I'm not happy with, figures that really need to be redrawn, and I've had almost no time to hunt down inconsistencies, typos, or errors. In other words, {\it this book is not finished}. If I didn't have a looming teaching deadline and a baby due in a few weeks, I really wouldn't be making this available at all. \vsp

What this means is that if you are an academic looking for teaching materials, a Ph.D. student looking to learn \R, or just a member of the general public interested in statistics, I would advise you to be cautious. What you're looking at is a first draft, and it may not serve your purposes. If we were living in the days when publishing was expensive and the internet wasn't around, I would never consider releasing a book in this form. The thought of someone shelling out \$80 for this (which is what a commercial publisher told me it would retail for when they offered to distribute it) makes me feel more than a little uncomfortable. However, it's the 21st century, so I can post the pdf on my website for free, and I can distribute hard copies via a print-on-demand service for less than half what a textbook publisher would charge. And so my guilt is assuaged, and I'm willing to share! With that in mind, you can obtain free soft copies and cheap hard copies online, from the following webpages:\vsp

\noindent
\begin{tabular}{ll}
Soft copy: &\url{http://www.compcogscisydney.com/learning-statistics-with-r.html}\\
Hard copy: & \url{www.lulu.com/content/13570633} 
\end{tabular}
\vsp

Even so, the warning still stands: what you are looking at is Version 0.3 of a work in progress. If and when it hits Version 1.0, I would be willing to stand behind the work and say, yes, this is a textbook that I would encourage other people to use. At that point, I'll probably start shamelessly flogging the thing on the internet and generally acting like a tool. But until that day comes, I'd like it to be made clear that I'm really ambivalent about the work as it stands. \vsp

All of the above being said, there is one group of people that I can enthusiastically endorse this book to: the psychology students taking our undergraduate research methods classes (DRIP and DRIP:A) in 2013. For you, this book is ideal, because it was written to accompany your stats lectures. If a problem arises due to a shortcoming of these notes, I can and will adapt content on the fly to fix that problem. Effectively, you've got a textbook written specifically for your classes, distributed for free (electronic copy) or at near-cost prices (hard copy). Better yet, the notes have been tested: Version 0.1 of these notes was used in the 2011 class, Version 0.2 was used in the 2012 class, and now you're looking at the new and improved Version 0.3. I'[for a historical summary]m not saying these notes are titanium plated awesomeness on a stick -- though if {\it you} wanted to say so on the student evaluation forms, then you're totally welcome to -- because they're not. But I am saying that they've been tried out in previous years and they seem to work okay. Besides, there's a group of us around to troubleshoot if any problems come up, and you can guarantee that at least {\it one} of your lecturers has read the whole thing cover to cover! \vsp

Okay, with all that out of the way, I should say something about what the book aims to be. At its core, it is an introductory statistics textbook pitched primarily at psychology students. As such, it covers the standard topics that you'd expect of such a book: study design, descriptive statistics, the theory of hypothesis testing, $t$-tests, $\chi^2$ tests, ANOVA and regression. However, there are also several chapters devoted to the \R\ statistical package, including a chapter on data manipulation and another one on scripts and programming. Moreover, when you look at the content presented in the book, you'll notice a lot of topics that are traditionally swept under the carpet when teaching statistics to psychology students. The Bayesian/frequentist divide is openly disussed in the probability chapter, and the disagreement between Neyman and Fisher about hypothesis testing makes an appearance. The difference between probability and density is discussed. A detailed treatment of Type I, II and III sums of squares for unbalanced factorial ANOVA is provided. And if you have a look in the Epilogue, it should be clear that my intention is to add a lot more advanced content.\vsp

My reasons for pursuing this approach are pretty simple: the students can handle it, and they even seem to enjoy it. Over the last few years I've been pleasantly surprised at just how little difficulty I've had in getting undergraduate psych students to learn \R. It's certainly not easy for them, and I've found I need to be a little charitable in setting marking standards, but they do eventually get there. Similarly, they don't seem to have a lot of problems tolerating ambiguity and complexity in presentation of statistical ideas, 
as long as they are assured that the assessment standards will be set in a fashion that is appropriate for them.  So if the students can handle it, why {\it not} teach it? The potential gains are pretty enticing. If they learn \R, the students get access to CRAN, which is perhaps the largest and most comprehensive library of statistical tools in existence. And if they learn about probability theory in detail, it's easier for them to switch from orthodox null hypothesis testing to Bayesian methods if they want to. Better yet, they learn data analysis skills that they can take to an employer without being dependent on expensive and proprietary software. \vsp

Sadly, this book isn't the silver bullet that makes all this possible. It's a work in progress, and maybe when it is finished it will be a useful tool. One among many, I would think. There are a number of other books that try to provide a basic introduction to statistics using \R, and I'm not arrogant enough to believe that mine is better. Still, I rather like the book, and maybe other people will find it useful, incomplete though it is.

\vspace*{24pt}
\noindent
Dan Navarro 

\noindent
January 13, 2013
%\today
\clearpage 
%\vspace*{36pt}











