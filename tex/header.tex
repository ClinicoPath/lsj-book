% !TEX root = ../pdf/lsr.tex
% [There are multiple lsr.tex files, but the one in ../pdf is the usual one]

\documentclass[a4paper,twoside,10pt,openright]{book}

% packages...
\usepackage{epsfig}
\usepackage{boxedminipage}
\usepackage{titlesec}
\usepackage{titletoc}
\usepackage{bm}
\usepackage{fancyhdr}
\usepackage{multicol}
\usepackage{alltt}
\usepackage[cmbtt]{bold-extra}
\usepackage{fancyvrb}
\usepackage{color}
\usepackage{verbatim}
\usepackage{amssymb}
\usepackage{multirow} 
\usepackage{morefloats}
\usepackage{dialogue}

% indexing package
%\usepackage{makeidx}
%\makeindex
%\newcommand{\ifn}[1]{\protect{#1}()@\protect\rtext{\protect{#1}()}}
%\newcommand{\ir}[1]{\protect{#1}@\protect\rtext{\protect{#1}}}
%\newcommand{\ir}[1]{#1@\rtext{\protect{#1}}}
%\usepackage{multind}
%\makeindex{R}
%\makeindex{ideas}

% provides url and URL commands (apparently has some nice interactions with
% hyperref for producing clickable links in pdf)
\usepackage{url}

% necessary on Windows/MiKTeX to stop pdfTeX whining about EPS files 
% (TeXShop on Mac seems to use the shell escape trick to convert on 
% the fly)
\usepackage{epstopdf}    

% set margins 
% ... A4: 297 x 210 mm
% ... cengage: 276mm x 210mm
% ... need extra: 1cm either side from the top and bottom?

\usepackage[left=2.5cm, right=2.5cm,top=4cm, bottom=4cm]{geometry}
%\usepackage[left=2.5cm, right=2.5cm,top=3cm, bottom=3cm]{geometry}

%\usepackage{fancybox}



\usepackage{afterpage}
\newcommand\blankpage{%
    \null
    \thispagestyle{empty}%
    \addtocounter{page}{-1}%
    \newpage}



\usepackage{mathabx} % provides the \bigoasterisk symbol


% -- comment out for the HARDCOPY version ---
% Need to  have everything loaded BEFORE hyperref, except for apacite,
% which needs to be loaded afterwards. 
\usepackage[pagebackref=true]{hyperref} % urls...
\hypersetup{colorlinks=true,
    citecolor={red},
    urlcolor={red},
    linkcolor={red},
	pdftitle={Learning Statistics with R},
	pdfauthor={Daniel Navarro},
}
%% http://health.webdev.adelaide.edu.au/psychology/ccs/docs/lsr/lsr.pdf#nameddest=chapter.17


% referencing
\usepackage{apacite}


%%%%%%%%%%%%%%%% define commands etc %%%%%%%%%%%%

% how are file names formatted?
\newcommand{\filename}[1]{\texttt{#1}}

% indicator that this is an advanced section
%\newcommand{\advanced}{$\bigoasterisk$}
\newcommand{\advanced}{}

\usepackage{manfnt}
%\newcommand{\advanced}{\protect\raisebox{9pt}{\small\dbend}}

% handy shortcuts
\newcommand{\given}{\,|\,} 
\newcommand{\panel}[1]{#1}  
\newcommand{\be}{\begin{equation}}
\newcommand{\ee}{\end{equation}}
\newcommand{\R}{\textsf{R}}

% what colour is the margin text?
%\definecolor{margintextcolour}{rgb}{.5,0,.5}
\definecolor{margintextcolour}{rgb}{.5,.5,.5}

% R code colours (outside of the rblock environments)
\definecolor{Rtextmar}{rgb}{.4,.4,.4}    % Rtext colour for margins

% define R code colours in the rblock environments
\definecolor{Rbarcol}{gray}{1.0}          % sidebar: was 0.8

% --- SOFTCOPY version ---
\definecolor{Rtextcol}{rgb}{0,0,0.5}    % Rtext colour in main text
\definecolor{Rboxcol}{rgb}{0.4,0.4,0.5}   % default Rblock colour (R output)
\definecolor{usercommand}{rgb}{0,0,.5}     % user command colour
\definecolor{Rscriptcol}{rgb}{.2,.2,.5} % R code colours inside scripts


%% --- HARDCOPY version ---
%\definecolor{Rboxcol}{rgb}{0.5,0.5,0.5}   % default Rblock colour (R output)
%\definecolor{usercommand}{rgb}{0,0,0}     % user command colour
%\definecolor{Rtextcol}{rgb}{0,0,0}    % Rtext colour
%\definecolor{Rscriptcol}{rgb}{.5,.5,.5} % R code colours inside scripts


% command to invoke user command formatting
\newcommand{\usr}[1]{\textcolor{usercommand}{#1}}
\newcommand{\plusorminus}{$\pm$}

% command to invoke help file code formatting
\newcommand{\helpcode}[1]{\texttt{\textcolor{black}{#1}}}

% define the "standard" rblock environment with
% no escape characters
\DefineVerbatimEnvironment{rblock}{Verbatim}{
    fontfamily=tt,
    frame=leftline,
    framesep=5mm,
    framerule=.25mm,
    formatcom=\color{Rboxcol},
    xleftmargin=2mm,
    xrightmargin=5mm,
    fontsize=\small,
    rulecolor=\color{Rbarcol}
}

% modified rblock that allows code insertion, using 
% @ as the escape character, and { and } as the block
% definition characters 
\DefineVerbatimEnvironment{rblock1}{Verbatim}{
    fontfamily=tt,
    frame=leftline,
    framesep=5mm,
    framerule=.25mm,
    formatcom=\color{Rboxcol},
    xleftmargin=2mm,
    xrightmargin=5mm,
    fontsize=\small,
    rulecolor=\color{Rbarcol},
    commandchars=\@\{\}
}

% modified rblock that allows code insertion, using 
% # as the escape character, and [ and ] as the block
% definition characters 
\DefineVerbatimEnvironment{rblock2}{Verbatim}{
    fontfamily=tt,
    frame=leftline,
    framesep=5mm,
    framerule=.25mm,
    formatcom=\color{Rboxcol},
    xleftmargin=2mm,
    xrightmargin=5mm,
    fontsize=\small,
    rulecolor=\color{Rbarcol},
    commandchars=\#[]
}

% modified rblock that allows code insertion, using 
% # as the escape character, and @ and @ as the block
% definition characters 
\DefineVerbatimEnvironment{rblock3}{Verbatim}{
    fontfamily=tt,
    frame=leftline,
    framesep=5mm,
    framerule=.25mm,
    formatcom=\color{Rboxcol},
    xleftmargin=2mm,
    xrightmargin=5mm,
    fontsize=\small,
    rulecolor=\color{Rbarcol},
    commandchars=\@QW
}

	
% verbatim environment(s) for help documentation
\DefineVerbatimEnvironment{rhelp}{Verbatim}{
    fontfamily=helvetica,
    frame=leftline,
    framesep=5mm,
    framerule=.25mm,
    formatcom=\color{Rboxcol},
    xleftmargin=2mm,
    xrightmargin=5mm,
    fontsize=\small,
    rulecolor=\color{Rbarcol},
}
\DefineVerbatimEnvironment{rhelp1}{Verbatim}{
    fontfamily=helvetica,
    frame=leftline,
    framesep=5mm,
    framerule=.25mm,
    formatcom=\color{Rboxcol},
    xleftmargin=2mm,
    xrightmargin=5mm,
    fontsize=\small,
    rulecolor=\color{Rbarcol},
    commandchars=\@\{\}
}
\DefineVerbatimEnvironment{rhelp2}{Verbatim}{
    fontfamily=tt,
    frame=leftline,
    framesep=5mm,
    framerule=.25mm,
    formatcom=\color{black},
    xleftmargin=2mm,
    xrightmargin=5mm,
    fontsize=\small,
    rulecolor=\color{Rbarcol}
}

% script environment 
\DefineVerbatimEnvironment{script}{Verbatim}{
    fontfamily=tt,
    frame=leftline,
    framesep=5mm,
    framerule=.25mm,
    numbers = left,
    formatcom=\color{Rscriptcol},
    xleftmargin=2mm,
    xrightmargin=5mm,
    fontsize=\small,
    rulecolor=\color{Rbarcol},
    commandchars=\@\|\_
}

% function for naming scripts
%\newcommand{\scriptname}[1]{\vspace*{6pt}\\ \small\texttt{ \# "#1.R"  \# }\vspace*{-3pt}}
\newcommand{\scriptname}[1]{\vspace*{6pt} \small\underline{\texttt{#1}}\vspace*{-3pt}}
\newcommand{\cmt}[1]{\textcolor{Rboxcol}{#1}} % script commment command


% define the rtext and rtextsmall commands. the small
% version is for use in margin paragraphs, so the font
% colour changes accordingly
\newcommand{\rtext}[1]
    {\texttt{\small\textcolor{Rtextcol}{#1}}}
    
\newcommand{\rtextoutput}[1]
    {\texttt{\small\textcolor{Rboxcol}{#1}}}
      
\newcommand{\rtextsmall}[1]
    {\texttt{\scriptsize\textcolor{Rtextmar}{#1}}}

% define rtextverb and rtextsmallverb
\CustomVerbatimCommand{\rtextverb}{Verb}{
	 fontsize = \small,
     fontfamily=tt,
     formatcom=\color{Rtextcol}
}

\CustomVerbatimCommand{\rtextsmallverb}{Verb}{
     fontfamily=tt,
     formatcom=\color{Rtextmar},
     fontsize=\scriptsize
}

% stop LaTeX from screwing the spacing
\raggedbottom

% define a \TODO command 
\newcommand{\TODO}{{\bf \{TODO\} }}

% define a command for the spacing between table caption and
% the actual table
\newcommand{\tabcapsep}{\vspace*{6pt}}

% command for standard errors
\newcommand{\SE}[1]{\mbox{\textsc{se}}(#1)}

% formatting for maths in footnotes...
\definecolor{bmgreycol}{gray}{.2}  
\newcommand{\bmgrey}[1]{\textcolor{bmgreycol}{\bm{#1}}}
\newcommand{\fngrey}[1]{\textcolor{bmgreycol}{#1}}
\newcommand{\FOOTNOTE}[1]{\footnote{\protect\fngrey{#1}}}

% formatting for matrix transpose
\newcommand{\T}{\prime}

% formatting line under figs
\newcommand{\dotrule}[1]{%
   \parbox[t]{#1}{\dotfill}}
\newcommand{\HR}{\vspace*{3pt}  \dotrule{1.0\textwidth}}

% --- SOFTCOPY version ---
\definecolor{keytermcol}{rgb}{.6,0,.8} % soft copy
\newcommand{\keyterm}[1]{{\bf \textcolor{keytermcol}{#1}}}

% --- HARDCOPY version ---
%\definecolor{keytermcol}{rgb}{0,0,0} % printed version
%\newcommand{\keyterm}[1]{\underline{\bf \textcolor{keytermcol}{#1}}}

% suppress footnote breakgin
\interfootnotelinepenalty=10000
